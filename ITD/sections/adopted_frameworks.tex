\section{Adopted Frameworks}

\subsection{Spring Boot}
Spring is a framework which provides comprehensive infrastructure support for developing Java applications and Spring Boot is basically an extension of Spring which eliminated the boilerplate configurations required for setting up a Spring application.
In other words, while the Spring framework focuses on providing flexibility to you, Spring Boot aims to shorten the code length and provide you with the easiest way to develop a web application. With annotation configuration and default codes, Spring Boot shortens the time involved in developing an application. It helps create a stand-alone application with less or almost zero-configuration.\\

\textbf{Why Spring Boot?}
\begin{itemize}
\item To ease the Java-based applications Development, Unit Test and Integration Test Process.
\item To reduce Development, Unit Test and Integration Test time by providing some defaults.
\item To increase Productivity.
\end{itemize}

\textbf{Why Spring?}
\begin{itemize}
\item Spring is a light weight framework and It minimally invasive development with POJO.
\item Spring achieves the loose coupling through dependency injection and interface based programming.
\item Spring supports declarative programming through aspects and common conventions.
\end{itemize}

\textbf{Inversion of Control}\\
Inversion of Control is a principle in software engineering by which the control of objects or portions of a program is transferred to a container or framework. It’s most often used in the context of object-oriented programming.

By contrast with traditional programming, in which our custom code makes calls to a library, IoC enables a framework to take control of the flow of a program and make calls to our custom code. To enable this, frameworks use abstractions with additional behavior built in. If we want to add our own behavior, we need to extend the classes of the framework or plugin our own classes.

The advantages of this architecture are:
\begin{itemize}
\item decoupling the execution of a task from its implementation
\item making it easier to switch between different implementations
\item greater modularity of a program
\item greater ease in testing a program by isolating a component or mocking its dependencies and allowing components to communicate through contracts
\end{itemize}

Inversion of Control can be achieved through various mechanisms such as: Strategy design pattern, Service Locator pattern, Factory pattern, and Dependency Injection (DI).

\subsection{AngularJS}

Designed by Google, AngularJS is an open-source framework that addresses the challenges of web development processes, it is a framework for internet applications, it is one of the most powerful front-end frameworks. 
Angular framework allows the use of HTML as the template language and allows the extension of HTML’s syntax to express application’s components in a brief and clear manner. It can automatically synchronize with models and views. Angular could literally be described as a hybrid HTML editable version that creates Apps.\\

\textbf{Why AngularJS?}
\begin{itemize}
\item Promotes Code Reusability: with Angular, developers can reuse the codes that were previously used in other different application, thus promoting code reusability and making Angular a unique and outstanding framework.
\item Faster application development: Angular makes everything from development to testing and maintenance quite fast and quick. The MVC (model view controller) architecture assures this.
\item Gives Developers full Control: developers can test, construct, inject and do almost anything with this framework, this is due to the fact that directives offer a free hand to experiment with HTML and attributes.
\item Data Binding: is easy with Angular, data binding happens with ease.
\item Time-Saving: all that Angular requires you to do is split your web app into multiple MVC(model view controller) components. Once that is done, Angular automatically takes over and performs the rest of the functions. It saves you from the trouble of writing another code.
\end{itemize}

\subsection{Ionic}
Ionic is a complete open-source SDK for hybrid mobile app originally built on top of AngularJS and Apache Cordova.
Ionic provides tools and services for developing hybrid mobile apps using Web technologies like CSS, HTML5, and Sass. Apps can be built with these Web technologies and then distributed through native app stores to be installed on devices by leveraging Cordova.
Ionic Framework is designed to work and display beautifully out-of-the-box across all platforms. It's a library of UI Components, which are reusable elements that serve as the building blocks for an application. Ionic Components are built with web standards using HTML, CSS, and JavaScript. Though the components are pre-built, they’re designed from the ground up to be highly customizable so apps can make each component their own, allowing each app to have its own look and feel. More specifically, Ionic components can be easily themed to globally change appearance across an entire app.\\

\textbf{Why Ionic?}
\begin{itemize}
\item Ionic is completely free and open source.
\item Ionic is built on Angular.
\item Ionic is a really “Native Like” framework.
\item Ionic has a beautiful default UI that is easy to customise.
\item It has a lot of tools and services.
\item Ionic integrates easily with native functionalities.
\end{itemize}

\subsection{Drawbacks}
Since the drawing up of the Design Document, we have been asking ourself if a hybrid client (with no local storage) for the third party could be a good choice or not. Now we can say that this choice requires few effort from the client point of view but it really reduces the performances of the server because it doesn't allow the server to exploit the caching capability of the client.
For simplicity, since we are developing a prototype, we have decided to continue with this choice but we have structured the third party front end code, in such a way to easily move this web browser app in a native desktop app, for Windows, MacOs, Linux. Of course this operation will require a change in the Design Document and in this document.
By choosing AngularJs as a front-end framework it will be much easier to do this transformation. Also the back-end will require just few changes, in particular the Storage Controller needs to retrieve data with timestamp greater than the timestamp of the last cached data on the client, and the ThirdParty Controller needs to be able to retrieve the timestamp of the last cached data on the client with REST API, call the proper method on the Storage Controller and then send the new data to the client.\\

\textbf{Electron}\\

Electron is a framework for creating desktop applications with all the emerging technologies including JavaScript, HTML and CSS. Basically, Electron framework takes care of the hard parts so that you can focus on the core of the application and revolutionize its design.
Designed as an open-source framework, Electron combines the best web technologies and is a cross-platform – meaning that it is easily compatible with Mac, Windows and Linux.\\

\textbf{Why Electron?}
\begin{itemize}
\item HTML, CSS, JS: Of course this is the most important point. It is amazing that you can now build Desktop Apps using these languages as it is very easy to learn and use them.
\item Electron Apps Are Similar To Web Apps: Part of what makes Electron Apps a good alternative to a native desktop app is the fact that Electron apps behave like Web Apps. What sets them apart is that Web Apps can only download files to the computer’s file system but Electron Apps can access the file system and can also read and write data.
\item Chromium: Electron uses Chromium engine for rendering UI. This means that you can get several benefits from this like Developer Tools, Storage Access, etc.
\end{itemize}

\subsection{Java Persistence API}
The Java Persistence API (JPA) is the Java standard for mapping Java objects to a relational database. Even though proprietary mapping products like Hibernate and TopLink still exist, they are now focused on providing their functionality through the JPA API, allowing all applications to be portable across JPA implementations
