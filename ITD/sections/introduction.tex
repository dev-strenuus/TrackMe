\section{Introduction}
\subsection{Purpose and Scope}


\subsection{Definitions, Acronyms, Abbreviations}

\subsubsection{Definitions}
\begin{itemize}
	\item \textbf{Framework}: is an abstraction in which software providing generic functionality can be selectively changed by additional user-written code.
	\item \textbf{Individual}:
	
	\item \textbf{Third party}:
	
	\item \textbf{User}:
	
\end{itemize}

\subsubsection{Acronyms}

\begin{center}
	\begin{tabular}{| l | l |}
		\hline
		DBMS & Data Base Management System\\
		HTTP & Hyper Text Transfer Protocol\\
		HTTPS & Hyper Text Transfer Protocol Secure\\
		API & Application Program Interface \\
		REST & REpresentational State Transfer\\
		MVC & Model View Controller\\
		JSON & JavaScript Object Notation \\
		\hline
	\end{tabular}
\end{center}

\subsection{Reference Documents}
\begin{itemize}
	\item Design document
	\item RASD document
	\item Project assignment
\end{itemize}

\subsection{Overview}
The rest of the document is organized in this way:
\begin{itemize}
	
	\item \textbf{Implemented requirements:} explains which functional requirements outlined in the RASD are accomplished, and how they are performed.
	\item \textbf{Adopted frameworks:} provides reasons about the implementation decisions taken in order to develop the application.
	\item \textbf{Source code structure:} explains and motivates how the source code is structured both in the front end and in the back end.
	\item \textbf{Testing:} provides the main testing cases applied to the the application
	
\end{itemize} 
