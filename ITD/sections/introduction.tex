\section{Introduction}
\subsection{Purpose and Scope}
	The following document represents the Implementation and Testing Document for
the TrackMe project.
The purpose of this document is to provide a comprehensive overview of the
implementation and testing activity for the development of the software application.
The TrackMe application aims at be a Health Data Management and Run-Friendly Mobile
App, useful for a quite wide range of users with different features and necessities.
\subsection{Definitions, Acronyms, Abbreviations}

\subsubsection{Definitions}
\begin{itemize}
	\item \textbf{Framework}: is an abstraction in which software providing generic functionality can be selectively changed by additional user-written code.
		
	\item \textbf{Third party}: a company, association or, more in general, a public or private entity that uses TrackMe to acquire data of users or to organize running events;

	\item \textbf{Individual}: a person who uses TrackMe;
	
	\item \textbf{JWT}: (or JSON Web Token) is a JSON-based open standard (RFC 7519) for creating access tokens.

\end{itemize}

\subsubsection{Acronyms}

\begin{center}
	\begin{tabular}{| l | l |}
		\hline
		DBMS & Data Base Management System\\
		HTTP & Hyper Text Transfer Protocol\\
		HTTPS & Hyper Text Transfer Protocol Secure\\
		API & Application Program Interface \\
		REST & REpresentational State Transfer\\
		MVC & Model View Controller\\
		JSON & JavaScript Object Notation \\
		VAT & Value Added Tax\\
		\hline
	\end{tabular}
\end{center}

\subsection{Reference Documents}
\begin{itemize}
	\item Design document
	\item RASD document
	\item Project assignment
\end{itemize}

\subsection{Overview}
The rest of the document is organized in this way:
\begin{itemize}
	
	\item \textbf{Implemented requirements:} explains which functional requirements outlined in the RASD are accomplished, and how they are performed.
	\item \textbf{Adopted frameworks:} provides reasons about the implementation decisions taken in order to develop the application.
	\item \textbf{Source code structure:} explains and motivates how the source code is structured both in the front end and in the back end.
	\item \textbf{Testing:} provides the main testing cases applied to the the application
	
\end{itemize} 
