\section{Drawbacks}
The main drawback of this propotype is the impossibility to continue to use the smartphone application after the raising of an SOS, even if the server continues to work. Even if the smartwatch simulator is a good idea, it makes really hard to connect and test simultaneously multiple individuals.\\
Another weakness of this project is that, although it's a prototype, several core functionalities are missing. For example, since the third party, as a client, has not been implemented, there is no way to see aggregate properties about an anonymous group. 
Moreover it's impossible to recognize a difference between a get of past data and a get of new data. Another missing functionality, not explicitly required but important for an application that manage sensitive data, is the possibility change the password and other personal data.\\
To conclude this section, it has to be noticed that anonimity is not really guaranteed for group requests. Let's focus on a particular case: a group request for all people with age between 22 and 23 years with subscription to new data activated. In this case let's imagine that at the moment of the request there are 900 people of 23 years old and 100 people of 22 years old and let's suppose that no people among 22-23 years old will sign up to the service in the next years. After a year, the group will change and it will be composed by only the 100 people that has changed their age from 22 to 23 years old (meanwhile the other 900 people are out of this group because they are 24 years old). It should be expected that no more data are retrieved from this group after a year, but this is not the outcome of this application.